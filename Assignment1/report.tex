\documentclass[letterpaper,12pt]{article}
\usepackage[margin=1in,letterpaper]{geometry}

\begin{document}

\title{Assignment 1}
\author{Tamal Deep Maity, R/No 20111068}
\date {September 25, 2020}
\maketitle


\section{Introduction}

A significant portion of this assignment deals with cleaning and extracting data. While doing it, some assumptions and valid conversions are made. These are important in the sense that, a lot of times when we try to map and match data obtained from different sources we find that the standards used by those sources vary and as such \verb|Data Integration| plays a significant role in such cases. Some of the cleaning and extraction policies are described below. 

To begin with the, the districts provided in the neighbor json had been assigned unique codes to them in the format \verb|districtName/Qx| where \verb|x| was a 7 digit number. We extracted the district name and their neighbors while keeping their mapping intact. Mapping districts between the json files \verb|data-all|  and   \verb|neighbor-districts|  was an interesting challenge. This stemmed from the fact that not all districts had the same name in both files (for eg. Balasore district in Odisha was named as Baleshwar) combined with that many districts had been since renamed from when the neighboring districts were mapped (for eg. Faizabad in UP was renamed as Ayodhya district). The first of the two issues mentioned above was handled by the use of the \verb|difflib| module in Python. This module contains a function \verb|get_close_matches()| that was used to track down the minor changes in district names. The latter problem involved much more manual effort that included but wasn't limited to district renaming searches in Wikipedia and State Government websites. To add on to the above challenges, some districts shared their names exactly with other districts (for eg. Aurangabad districts in Maharashtra as well as in Bihar). They were mapped by careful inspection of their neighboring districts. There were some states for which district data wasn't available, so we treated the entire state as a single district (5 such states were Telangana, Assam, Goa, Manipur and Sikkim) and mapped all neighbors of the state as the modified district's neighbors. We also deleted some districts for which Covid-19 data wasn't available and merged certain districts for which Covid-19 data was collected considering them as a single district . Once the matching was done a new json file was created called  \verb|neighbor-districts-modified|. This new json file had two distinct features : One, all districts (converted to lowercase) had been sorted alphabetically and two, they were given unique IDs to maintain identity. This ID was of the form : \verb|districtName_stateCode/x| where \verb|x| took on values starting from 101 based on lexicographical ordering. 

The modified neighbor district file was now considered as our base neighbor file and using the Covid-19 data from the other json file, problems 2 to 8 were coded and solved. While solving those, we defined some standards and made some assumptions. Some of those are listed below. 
\begin{itemize}
    \item Cases from states for which district data wasn't available were put under the modified district name as \verb|unknown_stateCode/ID|   
    \item Since we were asked to construct an undirected edge graph, an edge \verb|[i,j]| was included only when districts \verb|i| and \verb|j| were neighbors and \verb|i<j|.
    \item For districts with no neighbors, mean and standard deviation of number of cases in neighboring districts were computed as 0 and for districts with just 1 neighboring district, standard deviation was computed as 0. Similar policy was followed for computing mean and standard deviation of number of cases in all other districts of the state for a given district.
    \item The z-score was taken as 0 for the districts whose standard deviation(neighborhood/state) was coming out to be 0. 
\end{itemize}


\section{Analysis of the data}

It is particularly interesting to note the trend of Covid-19 cases in all districts of India. Some districts are worse hit than others and cases continue to rise there as in the case of Pune while some other districts seem to have passed the peak which is the case with Delhi. Delhi seemed to peak around a couple of months ago and since then the number of cases have come down every week. Districts like Bengaluru urban, Thane and Chennai have also seen huge load of cases during the observed phase. Also, some districts of North Eastern States like Kohima of Nagaland has negative Covid-19 cases for several weeks. This might be part of a correction procedure by the officials collecting the data. As analysts, we can't really make those numbers 0 because surely, they have been already included in the positive cases of some other districts on which we don't have any information.

One interesting observation can be made here with respect to the National Lockdown that was imposed all over the country effective from 25th March,2020. By the time the Unlock 1.0 started, the Centre had given States, the permission to impose lockdown rules and restrictions on their own. Now albeit, a lot of other factors come into play too, while considering the load of Covid-19 cases, the rules and restrictions put in place can't be overlooked completely. Take the example of Pune and Chennai. Around the time, when Lockdown was lifted by the Central government (around 30th of May), Chennai had around 1000 more confirmed cases than Pune. Now even though population of Chennai is nearly twice that of Pune, a couple of months later, Pune had 14000 more cases than Chennai.  A lot of other districts faced a similar issue of a steep rise in cases as well. It can be argued that districts which had significant number of cases, imposed stricter rules and thus they were able to restrict the growth to some extent. In this regard Delhi seems to have done better. Populated with 3 crore people, Delhi managed a significant dip in cases. However, it also showed us that such dips aren't irreversible. Cases rose by two times under three weeks after seeing a dip and that gives us a very accurate demonstration of population response once the threat of an infectious disease seems to have disappeared. A very similar trend was also noticed in yet another hotspot, Mumbai. After going through a dip in cases, cases doubled in under three weeks time.

The ever increasing cases in big cities also can't just be the result of a single factor. Another important factor to consider is travel resumption. Air travel was resumed around 25th May in almost all of India after a gap of 2 months. Cases in districts, especially in southern India, rose a whole lot since then. Bengaluru Urban coming out as an overall hotspot based on its neighbours is not very surprising. Neither is Chennai coming out as a state overall hotspot. These cities picked up cases quite late and since the lockdown already had been in place for a significant time, restrictions were getting lifted. Cases increased with increasing relaxations and this might have started a Domino effect of sorts.

Let us now shift our focus to districts or states that have done well with respect to Covid-19 cases. North Eastern states, barring Assam, take the centre stage here and it is very unsurprising. With a population density of 0.5\% of Mumbai, the fact that an infectious disease hasn't spread much in North Eastern states is easily describable and very much welcome. States of Arunachal Pradesh and Tripura thus have a number of districts (most of them bordering Assam) as coldspots . Infact, most of the districts bordering Assam, like East Jaintia Hills of Meghalaya or Kolasib of Mizoram developed as neighborhood coldspots owing to two reasons. One, that Assam has a significant case load and two, that since the district wise data of Assam wasn't available we were forced to consider the entire state as a single district which pushed the number of cases in the modified district by a humongous amount. However, these districts don't find a place in the state coldspots, as the number of cases in their entire respective states isn't significant enough to make them a coldspot. Districts that saw low cases even when other districts in the state had significant cases, are instead the state coldspots. Lahaul and Spiti of Himachal Pradesh and Wayanad of Kerala are some examples. 

Districts in Gujarat like Ahmedabad and Surat have significantly more cases than their neighbors. A part of the reason is that in cities like those the population is supremely higher than their neighbors. For example, in this case, these aforementioned cities are the two largest cities (population wise) in Gujarat. The next highest populated city in the state is Vadodara which only has a fourth of the population of Ahmedabad. So, even though the data for those districts seem very concerning, the population distribution is one major factor that comes into the picture here. As such, a more reliable index of an outburst of Covid-19 cases would be some sort of combination of the number of cases, the district population and the district population density. 

\section{Conclusion}

The Covid-19 suppression was never an easy task to begin with in one of the most populous and dense countries of the world. Public ignorance and lack of testing and tracing in early stages contributed hugely to the stranglehold this virus managed. But even though some cities, especially metros, have been hit hard, the data we analysed suggests that India could've found itself in a worse dire had the disease been allowed to spread to the villages and smaller cities. Some of the districts that are placed very precariously in this regard are Raigarh whose neighbors include high case load districts like Pune and Thane, Faridabad in Haryana sharing its borders with Gurugrama and Delhi, etc. Fortunately, this has been kept under some level of control so far. 486 out of the 627 districts we analysed, registered less than 5000 cases over the last 7 months. Although this data gives hope, the infection rate of this disease is seriously high and people and their governments can't afford to let their guards down, lest they forget about Pune, where cases rose upto 10 times in the matter of a month. Testing and tracing still holds the key in battling this virus. Hotspots have to be sealed quickly and efficiently so that the cases don't start trickling into neighbouring rural areas where medical facilities aren't ample. It is a difficult challenge in this era of increased networking and transporting needs and facilities but not an absolutely insurmountable task as well. 

\end{document}
