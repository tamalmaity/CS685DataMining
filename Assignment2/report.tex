\documentclass[letterpaper,12pt]{article}
\usepackage[margin=1in,letterpaper]{geometry}
\usepackage{hyperref}

\begin{document}

\title{Assignment 2}
\author{Tamal Deep Maity, R/No 20111068}
\date {November 20, 2020}
\maketitle


\section{Introduction}

This assignment is based on the dataset obtained from the human computation game,  \textbf{Wikispeedia}. The game is simple. We have to reach the destination article by clicking on the links available in the Wikipedia page starting from the source article. The heart of the dataset is thus the paths that have been taken while going or trying to go from the source to the destination article. As is often the case with sampled datasets, here too, we face some inconsistency in the dataset. We have thus, made some logical assumptions and proceeded with our code and analysis. Some of the assumptions and changes are mentioned below. 

\begin{itemize}
    \item For some articles, no category mapping was found. All such articles have been assumed to be of one single category of \verb|subjects|. 
    \item In \verb|paths_finished.tsv|, there are 11 paths that have a single node, i.e., the source and destination articles are same and as such there is no valid path. Those rows have been dropped along with another row (line number 2412) for which the shortest path was just mentioned as \verb|_|.
    \item Only for question.5 the graph made by utilising the data from  \verb|shortest-path-distance-| \verb|matrix.txt| was undirected, i.e., an edge from node A to node B, also means there exists an edge between node B to node A.
    \item All floating point numbers involved in output files have been rounded off to 4 decimal places.
\end{itemize}


\section{Analysis of the data}

The edges in the directed graph constructed signify the links between articles, i.e., an edge from article A to article B means that it is possible to visit the Wikipedia page for article A and go to article B's Wikipedia page from there by clicking on the link of that article. Now, the Wikipedia page for an article contains all sorts of relevant information for that article. It also contains analogies, antonyms, examples, and many other sorts of related content. So, one can very well understand that an edge, in a broad fashion, signifies that the articles are somehow related. Let us take the example of the article \verb|Batman|. Edges from this article go to articles like \verb|Chemistry| (related to Batman as the first ever Batman story was named "The Case of the Chemical Syndicate"), \verb|Crime| and \verb|Superman|. Edges to the article \verb|Batman| similarly, originate from \verb|Captain_Marvel|, \verb|Christian Bale| (an actor who played Batman) etc. 

The graph we formed has only 1\% edges as compared to the complete graph (edges between every two vertices) that could have been generated with the nodes(articles). This was as anticipated because it is very obvious that links to all articles can't be contained in a single Wikipedia page. The \href{https://dlab.epfl.ch/wikispeedia/play/}{website} hosting the \verb|Wikispeedia| game mentions that it is always possible to go from one article to another article. However, as we have mentioned above, there are some discrepancies in the dataset and as such we find 14 connected components. Here, connected component means that all the nodes in the graph can be reached from one another. So, even though there are some discrepancies, we would expect a huge chunk of the nodes(articles) to be reachable from one another and that too in not many clicks. This can be verified from the ouput of question.5 where we can see that of those 14 components, 12 of them contain just one node and there is one big connected component that accounts for almost all of the 4604 articles. The diameter of that connected component is 5 which means it is always possible to start with the Wikipedia page of an article from the big connected component and reach any other article with a maximum of 5 clicks. 

Of all the finished paths traversed, 17\% of the paths had back clicks. Now let us try to find out the logic behind the player clicking \verb|BACK|. Ideally a player would want to reach the destination as quick as possible, i.e., with minimum number of clicks. Now once a player identifies that the link they have pressed takes them to an article that is very little related to the destination article upon reading the page, they might consider going back. Of course, with human beings this game doesn't offer much resistance. To understand why this is the case, let us consider an example. Say the source article is \verb|London| and the destination article is \verb|Greece|. Now a human can, with some intuition, guess which links to click to minimise the number of clicks required to reach the destination. Seeing that the destination is a country, a very sensible choice is to click on the link of \verb|UK| in the wiki page for London. This takes them to the wiki page for UK. Now the human thinks along the lines of similarities that these two countries possess. The obvious one being that both belong to the European continent. A short read at the Wiki page suggests \verb|European Union| as a link. It would be very logical to click it and see what its member countries are and hope that \verb|Greece| is one of them. This is indeed the case and actually the shortest path to reach from \verb|London| to \verb|Greece|. As is obvious from the example above, the number of clicks can be kept below 5 (the diameter of the big connected component) most of the times, still it is also very much possible that the human playing the game has little previous information about the source and destination. In such cases it would be very difficult to choose the optimal path and hence a substantial 17\% of the paths had back clicks. Also, the average length of path in those back clicked paths only, was 10.62 which is easy interpretable as the human player would only use back clicks if he's getting further away from the destination in terms of relativity and using up more and more clicks. So, this bit of information also helps us reason that the other 83\% of paths without any back click should have a reasonably small value and close to 5. We find out that the average of those 83\% of paths is 4.72 which sits well with our reasoning.

\bigskip


The percentage of finished paths that use up the same number of clicks as the shortest possible path is slightly higher for paths that did not count back clicks against paths that counted back clicks. This is expected as it is very probable that the player who used back clicks paths knew lesser about the articles and thus was more prone to getting misdirected while trying to reach the destination. Thus not a lot of paths that used back clicks, which when were counted without back clicks gave the shortest path length. This accounts for the small difference in the percentages of the \verb|equal_length| column for the output files of question 7. After that column, for the remaining columns of paths that had difference more than 1 kept on decreasing till path difference of 10. This also means that not a lot of finished paths had a significant difference from the shortest path length. Since humans typically have some sort of knowledge about a range of topics, it is very rare that a person takes up lots of clicks to reach the destination. However, when we consider the subcategories count as well, these numbers (how many times or in how many paths the category was visited) ramp up because they have a lot of children (subcategories). Only articles \verb|Magdalena_Abakanowicz| and \verb|Niccol%C3%B2_dell%27Abbate| under the category of \verb|subject.Art.Artists| appear under no path. The number of paths in which the root node (\verb|subject|) was visited is equal for the human path and the shortest path because there is no such possibility of any path that doesn't contain the root category. Also, the ramp up of the numbers when subcategories count was considered as well, happened more distinctly for categories that had a number of subcategories that were traversed in the paths. This explains the high numbers for the categories at the root node or just after the level of the root node. 

\bigskip

As discussed above, the game typically depends not on how in depth the player knows about a certain topic but rather how much width does the player's knowledge domain contains. If a player knows the name of a Historian given as destination, he would very likely try to click on related articles like History, Famous People, Origin of the historian, etc. This is equivalent to traversing the breadth first tree that we created in question 2 in the most efficient way, i.e., try to go to the common ancestor of the tree first and then go down to the leaves where the article is typically mapped to a category. We can also see a lot of categories that had other subcategories within them don't have lots of articles under them as is obvious from the output files of question 8 and 9. In fact this makes practical sense. If we had an article named \verb|Sachin Tendulkar| we would want to map it to a category like \verb|subject.Indian.Famous,Cricketer| rather than just as \verb|subject| or \verb|subject.Indian|. This choice stems from the fact that we would want to narrow down our domain of guessing first and then rely on hit and trial rather than the other way around.

\bigskip

There are more number of category pairs that have all their paths as finished than category pairs that have all their paths as unfinished (almost twice). Now since we had almost twice the number of unfinished paths in our finished path file, this is consistent with our dataset. However, one interesting thing to note could be the destination categories corresponding to the paths that got finished the most and the paths that remained unfinished. The category for the former is \verb|subject.Language_and_literature.Languages| while for the latter is \verb|subject.Language_and_literature.General_Literature|. Although these topics are related, lots of players still did not manage to finish the path when the destination article was \verb|subject.Language_and_literature.General_Literature|. This might seem very weird but it in fact is not. A quick look at the output file \verb|category-paths.csv| for the number of times these categories were involved in a path, \verb|C0081| and \verb|C0080| respectively, tells us that category \verb|C0081| was almost 12 times as frequent as \verb|C0080| and hence even though the categories are related, their failure rate (destination not reached) is not. The output for question 11 tells us that almost 80\% of source-destination category pair took less than twice the shortest path possible. This depicts the fine ability of human players to intuitively guess the links that lead to the destination article quicker. It is however very rare (only 7\%) that a player can take exactly the same number of paths as the shortest path which signifies that though the player might have some intuition of what the shortest path could be, a machine can always outperform the player if such a game is played multiple times. 





\end{document}
